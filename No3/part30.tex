%!TEX root = ../NCVC6.tex

\mysection{読み込み条件を変えてGコードの生成}

\vspace*{1zh}
 1つ目の小細工ポイントです.
図\ref{fig:setup} のように切削レイヤも原点レイヤもG54に関する情報のみを読み込むように設定し[再読み込み]ボタンを押すと,
図\ref{fig:g54} のようにデータを絞ることができます.

\begin{figure}[H]
\centering
\includegraphics[scale=0.7]{No3/fig/setup.png}
\caption{CADデータの読み込み設定}
\label{fig:setup}
\end{figure}

\begin{figure}[H]
\centering
\includegraphics[scale=0.5]{No3/fig/g54.png}
\caption{G54に関するデータだけを読み込み}
\label{fig:g54}
\end{figure}

 この状態で普通にGコードを生成してください.
あとで手作業による修正があるので,カスタムヘッダーはそのままで結構です.
これをワーク座標系分繰り返します(このサンプルではG54からG57までの4回).

 ただし,それぞれのデータがわかるよう図\ref{fig:makedlg} のように出力ファイル名を変えてください.
上書きしてしまうとやり直しです.

\begin{figure}[H]
\centering
\includegraphics[scale=0.8]{No3/fig/makedlg.png}
\caption{Gコードの出力}
\label{fig:makedlg}
\end{figure}
